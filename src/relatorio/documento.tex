\documentclass{report}
\usepackage[T1]{fontenc} 
\usepackage[utf8]{inputenc} 
\usepackage[backend=biber, style=ieee]{biblatex} 
\usepackage{csquotes}
\usepackage[portuguese]{babel}
\usepackage{blindtext}
\usepackage[printonlyused]{acronym}
\usepackage{hyperref}
\usepackage{graphicx}
\usepackage{color}


\begin{document}
%
% Settings
%
\def\titulo{Proj2}
\def\data{12-06-2017}
\def\autores{Mário Liberato, Jorge Oliveira}
\def\autorescontactos{(84917) mliberato@ua.pt, (84983) jorge.am.oliveira@ua.pt}
\def\departamento{DETI}
\def\curso{MIECT}
\def\logotipo{ua.pdf}
%
%CAPA %
%
\begin{titlepage}

\begin{center}
%
\vspace*{50mm}
%
{\Huge \titulo}\\ 
%
\vspace{10mm}
%
{\Large \curso}\\
%
\vspace{10mm}
%
{\LARGE \autores}\\ 
%
\vspace{30mm}
%
\begin{figure}[h]
\center
\includegraphics{\logotipo}
\end{figure}
%
\vspace{30mm}
\end{center}
%
\end{titlepage}

% Pag Titulo %
\title{%
{\Huge\textbf{\titulo}}\\
{\Large \departamento\\ \curso}
}
%
\author{%
    \autores \\
    \autorescontactos
}
%
\date{\data}
%
\maketitle

\pagenumbering{roman}

%RESUMO%
\begin{abstract}

\end{abstract}

% Agradecimentos %
%\renewcommand{\abstractname}{Agradecimentos}
%\begin{abstract}
%\end{abstract}
%Não existem agradecimentos para este relatório

\tableofcontents
% \listoftables     
% \listoffigures    


%
\clearpage
\pagenumbering{arabic} %Numeracao fica a direita
%
\chapter{Introdução}
\label{chap.introducao} 

Neste projeto é pretendida a criação de uma aplicação móvel através da qual o utilizador possa enviar fotografias para um servidor, podendo ou não aplicar filtros. Também existe uma galeria contendo todas as imagens submetidas pelos grupos.

\chapter{Metodologia}
\label{chap.metodologia}

O grupo optou por utilizar Python 3 como linguagem de programação principal usufruindo do módulo Pillow e CherryPy. Para desenvolver o programa o grupo foi desenvolvendo um pouco de cada módulo até à sua finalidade.

\section{Testes}


\chapter{Descrição da aplicação}
\label{chap.desc}

\section{Página Web}

O objetivo do grupo era apresentar uma página simples e fácil de usar, usando o módulo Ratchet para compatibilidade móvel. Existiram protótipos em que devido a problemas de compatibilidade não foram implementados. Um exemplo seria a implementação de código javascript que mostrava alterações na imagem em "direto" à medida que o utilizador alterava os efeitos ou texto. 

Esta funcionalidade não foi implementada devido aos recursos utilizados no dispositivo tal como no servidor provocando grande lentidão.
\begin{figure}[h]
 \center
 \includegraphics[scale=0.5]{prototype.png}
 \caption{Protótipo da Página(Computador)}
 \label{ProtPag}
\end{figure}



\subsection{Página Inicial}

A página inicial tem carácter simples e de uso fácil. O utilizador pode selecionar uma imagem a qual é apresentada na própria página, e escolher os efeitos, ou o gerador de meme onde aplica texto e/ou um fundo à imagem. A letra da imagem ainda pode ser sombreada.

\subsection{Submissão de Imagem}

A submissão de imagens é feita através de um método /api/put, o qual usa um pedido POST que envia a imagem e os argumentos (com os efeitos a utilizar, por exemplo). Este método faz uso de um outro método na classe API para obter um ID para a imagem a carregar antes de o fazer.

\subsection{Galeria}

A página da galeria é apresentada de forma simples, também permite o utilizador realizar votos nas imagens como consultar os mesmos. A apresentação da galeria de imagens usa o método /api/listAll onde é fornecida uma lista com todas imagens dos grupos, votos e autor. Os votos são realizados através de /api/vote onde são enviados o identificador da fotografia, do utilizador e o tipo de voto.

\section{Núcleo da Aplicação}

\subsection{Efeitos}

Para a realização dos efeitos foi utilizado Python3 e o módulo Pillow.
Os efeitos disponíveis no programa são: \textit{blur}, escala de cinzentos, \textit{lomography}, sepia e invesão de cor. Para a realização dos dois primeiros a imagem é processada com desfocagem e alterado o modo da imagem. Para os outros efeitos são aplicadas fórmulas matemáticas aos canais Vermelho, Verde e Azul para fazer modificações pixel a pixel à imagem.

\subsection{Gerador de Memes}

Este módulo permite aplicar texto à imagem tanto na parte inferior como superior, sombreado ou não. Também tem como objetivo recortar uma cara da imagem e por sobre um fundo branco ou colorido, à escolha do utilizador. No entanto, este módulo causou algum transtorno vistos que não foi possível existir uma forma correta de recortar uma cara de uma imagem.

\subsection{Persistência}

\subsection{Autenticação}

\chapter{Resultados e Análise}
\label{chap.res}

\section{Página Web}

\section{Núcleo}


\chapter{Conclusões}
\label{chap.conc}

\chapter*{Contribuições dos autores}

ML trabalhou maioritariamente na backend da aplicação, realizando o servidor e a base dados tal como a adaptação dos efeitos ao servidor.

JO trabalhou maioritariamente na frontend da aplicação, dando a face ao programa nas páginas web, tal como o gerador de meme com seus efeitos.

\chapter*{Acrónimos}
\begin{acronym}
 \acro{api}[API] {Application Programming Interface/Interface de Programação de Aplicações}
 \acro{deti}[DETI]{Departamento de Electrónica, Telecomunicações e Informática}
 \acro{miect}[MIECT]{Mestrado Integrado em Engenharia de Computadores e Telemática}
 \acro{ua}[UA]{Universidade de Aveiro}
\end{acronym}


%
%\printbibliography

\end{document}
