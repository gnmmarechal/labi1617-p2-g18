\documentclass{report}
\usepackage[T1]{fontenc} 
\usepackage[utf8]{inputenc} 
\usepackage[backend=biber, style=ieee]{biblatex} 
\usepackage{csquotes}
\usepackage[portuguese]{babel}
\usepackage{blindtext}
\usepackage[printonlyused]{acronym}
\usepackage{hyperref}
\usepackage{graphicx}
\usepackage{color}


\begin{document}
%
% Settings
%
\def\titulo{Proj2}
\def\data{12-06-2017}
\def\autores{Mário Liberato, Jorge Oliveira}
\def\autorescontactos{(84917) mliberato@ua.pt, (84983) jorge.am.oliveira@ua.pt}
\def\departamento{DETI}
\def\curso{MIECT}
\def\logotipo{ua.pdf}
%
%CAPA %
%
\begin{titlepage}

\begin{center}
%
\vspace*{50mm}
%
{\Huge \titulo}\\ 
%
\vspace{10mm}
%
{\Large \curso}\\
%
\vspace{10mm}
%
{\LARGE \autores}\\ 
%
\vspace{30mm}
%
\begin{figure}[h]
\center
\includegraphics{\logotipo}
\end{figure}
%
\vspace{30mm}
\end{center}
%
\end{titlepage}

% Pag Titulo %
\title{%
{\Huge\textbf{\titulo}}\\
{\Large \departamento\\ \curso}
}
%
\author{%
    \autores \\
    \autorescontactos
}
%
\date{\data}
%
\maketitle

\pagenumbering{roman}

%RESUMO%
\begin{abstract}

\end{abstract}

% Agradecimentos %
%\renewcommand{\abstractname}{Agradecimentos}
%\begin{abstract}
%\end{abstract}
%Não existem agradecimentos para este relatório

\tableofcontents
% \listoftables     
% \listoffigures    


%
\clearpage
\pagenumbering{arabic} %Numeracao fica a direita
%
\chapter{Introdução}
\label{chap.introducao} 

Neste projeto é pretendida a criação de uma aplicação móvel através da qual o utilizador possa enviar fotografias para um servidor, podendo ou não aplicar filtros. Também existe uma galeria contendo todas as imagens submetidas pelos grupos.

\chapter{Metodologia}
\label{chap.metodologia}

O grupo optou por utilizar Python 3 como linguagem de programação principal usufruindo do módulo Pillow e CherryPy. Para desenvolver o programa o grupo foi desenvolvendo um pouco de cada módulo até à sua finalidade.

\section{Testes}


\chapter{Descrição da aplicação}
\label{chap.desc}

\section{Página Web}


\subsection{Página Inicial}


\subsection{Submissão de Imagem}

A submissão de imagens é feita através de um método /api/put, o qual usa um pedido POST que envia a imagem e os argumentos (com os efeitos a utilizar, por exemplo). Este método faz uso de um outro método na classe API para obter um ID para a imagem a carregar antes de o fazer.

\subsection{Galeria}

\section{Núcleo da Aplicação}

\subsection{Efeitos}

\subsection{Gerador de Memes}

\subsection{Utilização de API}

\subsection{Persistência}

\subsection{Autenticação}

\chapter{Resultados e Análise}
\label{chap.res}

\section{Página Web}

\section{Núcleo}


\chapter{Conclusões}
\label{chap.conc}



\chapter*{Acrónimos}
\begin{acronym}
 \acro{api}[API] {Application Programming Interface/Interface de Programação de Aplicações}
 \acro{deti}[DETI]{Departamento de Electrónica, Telecomunicações e Informática}
 \acro{miect}[MIECT]{Mestrado Integrado em Engenharia de Computadores e Telemática}
 \acro{ua}[UA]{Universidade de Aveiro}
\end{acronym}


%
%\printbibliography

\end{document}
